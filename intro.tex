\section{Einleitung}

Die steigenden Sicherheitsanforderungen von öffentliche Plätzen, kritischer Infrastruktur oder privater Grundstücken fordern oft kontinuierliche Videoüberwachung vieler Lokationen. Diese gewaltige Menge an Daten muss in Echtzeit ausgewertet werden, um sofort auf Gefahren reagieren zu können, welches in der Vergangenheit nur durch menschliche Operatoren möglich war. Eine Automatisierung dieser Aufgabe verringert die Kosten und steigert gleichzeitig die Zuverlässigkeit, weshalb Videosequenzanalyse und Mustererkennung immer mehr an Bedeutung gewinnen. Das Ziel ist es komplexe Aktivitäten und Akteure in einer Videoaufnahme zu erkennen und kategorisieren.

Solche Analysen sind oft hierarchisch aufgebaut, wobei zuerst Analysen auf niedriger Ebene durchgeführt werden, wie zum Beispiel Vorder- und Hintergrundanalyse \cite{CSEG:Tracking}. Hierbei werden die erwarteten Elemente des Bildes (der Hintergrund) von den unerwarteten (Vordergrund) getrennt.  

Viele Systeme, welche menschliche Aktivität erkennen, arbeiten ausschließlich mit visuellen Daten, aber andere Modalitäten wie Audio sind oft zusätzlich vorhanden und können genutzt werden um Aktivitätsmuster genauer zu erkennen.