\section{Bewertung} 

Die AVC Matrix Methode ist eine neue Methode um audiovisuelle Eventerkennung für Videoaufnahmen mit einer Kamera und einem Mikrofon zu automatisieren. Die Video- und Audiosignale werden separat mit adaptiven Modellen verarbeitet, um Vordergrund und Hintergrund zu unterscheiden. Anschließend werden anhand ihrer zeitlichen Synchronität Audio-Video-Events erkannt. Dies wird mit einer Audio-Video-Concurrency (AVC) Matrix berechnet. Die Methode lieferte in den durchgeführten Experimente bessere Ergebnisse, als die Verwendung reiner Audio- oder Videodaten. Auch liefert sie bessere Ergebnisse als die Verwendung von Audio-Video-Daten welche ohne die zeitliche Synchronität kombiniert wurden.

Die Methode verwendet simple Algorithmen für die Videoerkennung, wodurch zum Beispiel mehrere gleichfarbige Objekte nicht unterschieden werden können. Die AVC Matrix Methode kann jedoch auch mit weiterentwickelten Videoanalyse Methoden kombiniert werden. Außerdem ist diese Methode nicht in der Lage mehrere gleichzeitig erscheinende Events korrekt zu trennen.